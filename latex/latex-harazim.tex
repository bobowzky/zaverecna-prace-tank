% ŠABLONA PRO PSANÍ ZÁVĚREČNÉ STUDIJNÍ PRÁCE
%%%%%%%%%%%%%%%%%%%%%%%%%%%%%%%%%%%%%%%%%%%%
% Autor: Jakub Dokulil (kubadokulil99@gmail.com)
% Tato šablona byla vytvořena tak, aby pomocí ní mohli v systému LaTeX soutěžící sázet své práce a zároveň odpovídala požadavkům na formátování vyplývajícím z wordové šablony umístěné na webu soc.cz.
%
\documentclass[12pt, a4paper,
%oneside,      %% -- odkomentujte, pokud chcete svou práci mít pouze jednostrannou, mezera pro hřbet pak automaticky bude pouze na levé straně
twoside,        %% -- pro oboustranné práce, mezera pro hřbet následně střídá strany.
openany
]{report}

%% Nutné balíčky a nastavení
%%%%%%%%%%%%%%%%%%%%%%%%%%%%

%% Proměnné
\newcommand\obor{INFORMAČNÍ TECHNOLOGIE} %% -- napiš číslo a název tvého oboru
\newcommand\kodOboru{18-20-M/01} %% -- napiš číslo a název tvého oboru
\newcommand\zamereni{se zaměřením na počítačové sítě a programování} %% -- napiš číslo a název tvého oboru
\newcommand\skola{Střední škola průmyslová a umělecká, Opava} %% vyplň název školy
\newcommand\trida{IT4} %% vyplň jméno svého konzultanta
\newcommand\jmenoAutora{Robin Harazim}  %% vyplň své jméno
\newcommand\skolniRok{2023/24} %% vyplň rok
\newcommand\datumOdevzdani{1. 1. 2024} %% vyplň rok
\newcommand\nazevPrace{Tank na dálkové ovládání TM-1} %% vyplň název své práce

\title{\nazevPrace} %% -- Název tvé práce
\author{\jmenoAutora} %% -- tvé jméno
\date{\datumOdevzdani} %% -- rok, kdy píšeš SOČku

\usepackage[top=2.5cm, bottom=2.5cm, left=3.5cm, right=1.5cm]{geometry} %% nastaví okraje, left -- vnitřní okraj, right -- vnější okraj

\usepackage[czech]{babel} %% balík babel pro sazbu v češtině
\usepackage[utf8]{inputenc} %% balíky pro kódování textu
\usepackage[T1]{fontenc}
\usepackage{cmap} %% balíček zajišťující, že vytvořené PDF bude prohledávatelné a kopírovatelné

\usepackage{graphicx} %% balík pro vkládání obrázků

\usepackage{subcaption} %% balíček pro vkládání podobrázků

\usepackage{hyperref} %% balíček, který v PDF vytváří odkazy

\linespread{1.5} %% řádkování
\setlength{\parskip}{0.5em} %% odsazení mezi odstavci


\usepackage[pagestyles]{titlesec} %% balíček pro úpravu stylu kapitol a sekcí
\titleformat{\chapter}[block]{\scshape\bfseries\LARGE}{\thechapter}{10pt}{\vspace{0pt}}[\vspace{-22pt}]
\titleformat{\section}[block]{\scshape\bfseries\Large}{\thesection}{10pt}{\vspace{0pt}}
\titleformat{\subsection}[block]{\bfseries\large}{\thesubsection}{10pt}{\vspace{0pt}}


\usepackage{tocloft} % Balíček umožní přizpůsobit vzhled tabulky obsahu
\setlength{\cftbeforechapskip}{0pt}  % Menší rozestup pro kapitoly
\setlength{\cftbeforesecskip}{0pt}   % Menší rozestup pro sekce

\setcounter{secnumdepth}{2}
\setcounter{tocdepth}{3}
\usepackage{fancyhdr}
\pagestyle{fancy}
\renewcommand{\headrulewidth}{0.025pt}

\usepackage{booktabs}

\usepackage{url}

%% Balíčky co se můžou hodit :) 
%%%%%%%%%%%%%%%%%%%%%%%%%%%%%%%

\usepackage{pdfpages} %% Balíček umožňující vkládat stránky z PDF souborů, 

\usepackage{upgreek} %% Balíček pro sazbu stojatých řeckých písmen, třeba u jednotky mikrometr. Například stojaté mí: \upmu, stojaté pí: \uppi

\usepackage{amsmath}    %% Balíčky amsmath a amsfonts 
\usepackage{amsfonts}   %% pro sazbu matematických symbolů
\usepackage{esint}     %% pro sazbu různých integrálů (např \oiint)
\usepackage{mathrsfs}
\usepackage{helvet} % Helvet font
\usepackage{mathptmx} % Times New Roman
\usepackage{Oswald} % Oswald font


%% makra pro sazbu matematiky
\newcommand{\dif}{\mathrm{d}} %% makro pro sazbu diferenciálu, místo toho
%% abych musel psát '\mathrm{d}' mi stačí napsat '\dif' což je mnohem 
%% kratší a mohu si tak usnadnit práci

\usepackage{listings}
\usepackage{xcolor}

\renewcommand{\lstlistingname}{Kód}% Listing -> Algorithm
\renewcommand{\lstlistlistingname}{Seznam programových kódů}% List of Listings -> List of Algorithms

%% Definice 
\lstdefinelanguage{JavaScript}{
	morekeywords=[1]{break, continue, delete, else, for, function, if, in,
		new, return, this, typeof, var, void, while, with},
	% Literals, primitive types, and reference types.
	morekeywords=[2]{false, null, true, boolean, number, undefined,
		Array, Boolean, Date, Math, Number, String, Object},
	% Built-ins.
	morekeywords=[3]{eval, parseInt, parseFloat, escape, unescape},
	sensitive,
	morecomment=[s]{/*}{*/},
	morecomment=[l]//,
	morecomment=[s]{/**}{*/}, % JavaDoc style comments
	morestring=[b]',
	morestring=[b]"
}[keywords, comments, strings]


\lstdefinelanguage[ECMAScript2015]{JavaScript}[]{JavaScript}{
	morekeywords=[1]{await, async, case, catch, class, const, default, do,
		enum, export, extends, finally, from, implements, import, instanceof,
		let, static, super, switch, throw, try},
	morestring=[b]` % Interpolation strings.
}

\lstalias[]{ES6}[ECMAScript2015]{JavaScript}

% Nastavení barev
% Requires package: color.
\definecolor{mediumgray}{rgb}{0.3, 0.4, 0.4}
\definecolor{mediumblue}{rgb}{0.0, 0.0, 0.8}
\definecolor{forestgreen}{rgb}{0.13, 0.55, 0.13}
\definecolor{darkviolet}{rgb}{0.58, 0.0, 0.83}
\definecolor{royalblue}{rgb}{0.25, 0.41, 0.88}
\definecolor{crimson}{rgb}{0.86, 0.8, 0.24}

% Nastavení pro Python
\lstdefinestyle{Python}{
	language=Python,
	backgroundcolor=\color{white},
	basicstyle=\ttfamily,
	breakatwhitespace=false,
	breaklines=false,
	captionpos=b,
	columns=fullflexible,
	commentstyle=\color{mediumgray}\upshape,
	emph={},
	emphstyle=\color{crimson},
	extendedchars=true,  % requires inputenc
	fontadjust=true,
	frame=single,
	identifierstyle=\color{black},
	keepspaces=true,
	keywordstyle=\color{mediumblue},
	keywordstyle={[2]\color{darkviolet}},
	keywordstyle={[3]\color{royalblue}},
	literate=%
	{á}{{\'a}}1 {č}{{\v{c}}}1 {ď}{{\v{d}}}1 {é}{{\'e}}1 {ě}{{\v{e}}}1
	{í}{{\'i}}1 {ň}{{\v{n}}}1 {ó}{{\'o}}1 {ř}{{\v{r}}}1 {š}{{\v{s}}}1
	{ť}{{\v{t}}}1 {ú}{{\'u}}1 {ů}{{\r{u}}}1 {ý}{{\'y}}1 {ž}{{\v{z}}}1,		
	numbers=left,
	numbersep=5pt,
	numberstyle=\tiny\color{black},
	rulecolor=\color{black},
	showlines=true,
	showspaces=false,
	showstringspaces=false,
	showtabs=false,
	stringstyle=\color{forestgreen},
	tabsize=2,
	title=\lstname,
	upquote=true  % requires textcomp	
}


\lstdefinestyle{JSES6Base}{
	backgroundcolor=\color{white},
	basicstyle=\ttfamily,
	breakatwhitespace=false,
	breaklines=false,
	captionpos=b,
	columns=fullflexible,
	commentstyle=\color{mediumgray}\upshape,
	emph={},
	emphstyle=\color{crimson},
	extendedchars=true,  % requires inputenc
	fontadjust=true,
	frame=single,
	identifierstyle=\color{black},
	keepspaces=true,
	keywordstyle=\color{mediumblue},
	keywordstyle={[2]\color{darkviolet}},
	keywordstyle={[3]\color{royalblue}},
 literate=%
{á}{{\'a}}1 {č}{{\v{c}}}1 {ď}{{\v{d}}}1 {é}{{\'e}}1 {ě}{{\v{e}}}1
{í}{{\'i}}1 {ň}{{\v{n}}}1 {ó}{{\'o}}1 {ř}{{\v{r}}}1 {š}{{\v{s}}}1
{ť}{{\v{t}}}1 {ú}{{\'u}}1 {ů}{{\r{u}}}1 {ý}{{\'y}}1 {ž}{{\v{z}}}1,		
	numbers=left,
	numbersep=5pt,
	numberstyle=\tiny\color{black},
	rulecolor=\color{black},
	showlines=true,
	showspaces=false,
	showstringspaces=false,
	showtabs=false,
	stringstyle=\color{forestgreen},
	tabsize=2,
	title=\lstname,
	upquote=true  % requires textcomp
}

\lstdefinestyle{JavaScript}{
	language=JavaScript,
	style=JSES6Base,
}
\lstdefinestyle{ES6}{
	language=ES6,
	style=JSES6Base
}


%% Bordel pro práci - můžeš smáznout :) 
%%%%%%%%%%%%%%%%%%%

\usepackage{lipsum} %% balíček který píše lipsum (nesmyslný text, který se používá pro kontrolu typografie)

%% Začátek dokumentu
%%%%%%%%%%%%%%%%%%%%
\begin{document}
	
	\cleardoublepage
    \setcounter{page}{1}
    \pagestyle{empty}

%% Titulní stránka s informacemi
%%%%%%%%%%%%%%%%%%%%%%%%%%%%%%%%%%%%%%%%
	
	{\fontfamily{phv}\selectfont
		%% Logo školy
		\begin{figure}[h]
			\centering
			\includegraphics[width=0.6\linewidth]{image/logo-skoly.png} 
		\end{figure}
		
		
		%% Hlavička práce a její název (viz proměnná \nazev prace)
		%% \sffamily %%% bezpatkové písmo - sans serif
		{\bfseries %%% písmo na stránce je tučně
			\begin{center}
				\vspace{0.025 \textheight}
				\LARGE{ZÁVĚREČNÁ STUDIJNÍ PRÁCE}\\
				\large{dokumentace}\\
				\vspace{0.075 \textheight}
				\LARGE {\nazevPrace}\\
			\end{center}  
		}%%%
		
		\begin{figure}[h]
			\centering
			\includegraphics[width=0.8\linewidth]{image/programovani-02.jpg} 
		\end{figure}
		
		\vspace{0.02 \textheight}
		\begin{table}[h!]
			\begin{tabular}{ll}
				\textbf{Autor:} & \jmenoAutora\\ 
				\textbf{Obor:} & \kodOboru { } \obor\\
				\textbf{} & \zamereni\\
				\textbf{Třída:} & \trida\\
				\textbf{Školní rok:} & \skolniRok\\
			\end{tabular}
			
		\end{table}		
	}
	
\newpage 
	
%% Stránka obsahující poděkování a prohlášení
%%%%%%%%%%%%%%%%%%%%%%%%%%%%%%%%%%%%%%%%%%%%%%%%%%%%%%%%

%% Poděkování - nepovinné
%%%%%%%%%%%%%%%%%%%%%%%%%%%%
	\noindent{\large{\bfseries{Poděkování}\\}}
	\noindent Chtěl bych  poděkovat panu učiteli Mgr. Marcelu Godovskému za poskytnutí 3D tisku a užitečné rady při vytváření projektu.
	
	\vspace*{0.6\textheight} %% Vertikální mezeru je možné upravit

%% Prohlášení - povinné
%%%%%%%%%%%%%%%%%%%%%%%%%%%%
	\noindent{\large{\bfseries{Prohlášení}\\}}  %% uprav si koncovky podle toho na jaký rod se cítíš, vypadá to pak lépe :) 
	\noindent{Prohlašuji, že jsem závěrečnou práci vypracoval samostatně a uvedl veškeré použité 
		informační zdroje.\\}
	\noindent{Souhlasím, aby tato studijní práce byla použita k výukovým a prezentačním účelům na Střední průmyslové a umělecké škole v Opavě, Praskova 399/8.}
	\vfill
	\noindent{V Opavě \datumOdevzdani\\}
	\noindent
	\begin{minipage}{\linewidth}
		\hspace{9.5cm} 
		\begin{tabular}{@{}p{6cm}@{}}
			\dotfill \\
			Podpis autora
		\end{tabular}
	\end{minipage}
	
	\clearpage %% Zalomení dvojstránky

%% Stránka obsahující abstrakt (anotaci)
%%%%%%%%%%%%%%%%%%%%%%%%%%%%%%%%%%%%%%%%%%%%%%%%%%%%%%%%	

%% Abstrakt v češtině
%%%%%%%%%%%%%%%%%%%%%%%%%%%%
	\noindent{\Large{\bfseries{Abstrakt}\\}}
	\noindent Tato práce se zaměřuje na konstrukci dálkově ovládaného tanku, jehož většina součástek byla vyrobena technologií 3D tisku. Základní řídící jednotkou je vývojová deska ESP32 s vestavěnou kamerou a WiFi modulem, která umožňuje živé ovládání tanku prostřednictvím webového serveru postaveného na knihovně ESPAsyncWebServer. Webové rozhraní dává uživatelům možnost sledovat tank v reálném čase, což značně usnadňuje jeho ovladatelnost. Součástí projektu je také sestrojení vlastního ovladače jako možnost alternativního ovládání. \\
	
	\vspace{18pt}
	
	\noindent{\large{\bfseries{Klíčová slova}}}
	
	\noindent ESP32, Tank, závěrečná práce, dokumentace, \dots 
	
	
	\clearpage %% Zalomení stránky

%% Stránka s generovaným obsahem
%%%%%%%%%%%%%%%%%%%%%%%%%%%%%%%%%%%%%%%

% Save the original 'plain' page style
\fancypagestyle{originalplain}{
  \fancyhf{}
  \fancyfoot[C]{\thepage} % Example: put page number at center of footer.
  \renewcommand{\headrulewidth}{0pt}
  \renewcommand{\footrulewidth}{0pt}
}

% Redefine the 'plain' page style to be empty
\fancypagestyle{plain}{
  \fancyhf{} % clear all header and footer fields
  \renewcommand{\headrulewidth}{0pt}
  \renewcommand{\footrulewidth}{0pt}
}

\pagestyle{empty} % apply empty page style
	\tableofcontents %% Vygeneruje tabulku s obsahem
 
%% Stránka s úvodem - povinná část
%%%%%%%%%%%%%%%%%%%%%%%%%%%%%%%%%%%%%%%	
	\chapter*{Úvod}
	\addcontentsline{toc}{chapter}{Úvod}
 \fancypagestyle{plain}{%
  % Restore the original 'plain' style definitions
  \fancyhf{}
  \fancyfoot[C]{\thepage} % Example: put page number at center of footer.
  \renewcommand{\headrulewidth}{0pt}
  \renewcommand{\footrulewidth}{0pt}
}
    \pagestyle{plain}
 
Mou hlavní motivací pro tento projekt byla má dlouhodobá záliba ve vojenské technice, jenž se se mnou táhne již od dětství a také zájem v oblasti hardware. Rozhodl jsem se tedy tyto své dvě záliby zkombinovat a vytvořit tak tento projekt. \\
Cílem této práce bylo tedy vytvořit tank na dálkové ovládání s možností ovládání přes webový server nebo přes vlastnoručně vyrobený ovladač.
Při vytváření mého projektu jsem se nejprve musel rozhodnout, které součástky budou nejvhodnější, a zároveň jsem pečlivě promýšlel, zda si vybrat předem sestavený model nebo se pustit do výroby vlastních komponentů a jejich následného 3D tisku.\\
Původně jsem zamýšlel všechny díly sám vymodelovat, ale vzhledem k tomu, že toto řešení by bylo docela dost časově náročné, rozhodl jsem se najít si na internetu vhodný model podvozku i s pásy a zbytek domodelovat sám. Všechny díly jsem nakonec vytiskl na 3D tiskárně a smontoval dohromady.\\
Jako jádro projektu jsem si zvolil právě vývojovou desku ESP32, vybavenou integrovaným Wi-Fi modulem a kamerou, a jako první jsem chtěl vytvořit jednoduchý webový server pomocí knihovny ESPAsynchWebServer, který by umožňoval připojenému uživateli tank ovládat pomocí živého obrazu z kamery na serveru. Další problematikou by bylo sestrojení vlastního ovladače jako alternativní možnost pro ovládání tanku.\\
V této dokumentaci se tedy snažím podrobněji popsat svůj postup při řešení projektu ale také problémy se kterými jsem se setkal a jejich řešení. 

\chapter{Využité technologie}

\section{Hardware}
\label{sec:zakladni_struktura}

V této kapitole se podrobněji podíváme na jednotlivé součástky, které jsem ve svém projektu využil.

\subsection{Seznam součástek}

\begin{itemize}
  \item Vývojová deska ESP32-CAM, WiFi + Bluetooth
  \item H-můstek L298N, dvoumotorový modul
  \item 2x Plastové micro servo SG90 9g, 180°
  \item Součástky vytvořené pomocí 3D tisku
  \item Napájecí modul
  \item Nepájivé pole
\end{itemize}

\subsection{ESP}
Hlavní tělo dokumentu začíná příkazem \verb|\begin{document}| a končí \verb|\end{document}|. Veškerý obsah, který chcete mít ve svém dokumentu, by měl být umístěn mezi tyto dva příkazy. 

\subsection{Můstek}
Pro organizaci obsahu se často používají sekce a podsekce. Tyto struktury pomáhají čtenáři lépe navigovat dokumentem a rozdělit text do logických bloků.

\begin{verbatim}
	\section{Název sekce}
	\subsection{Název podsekce}
	\subsubsection{Název podpodsekce}
\end{verbatim}

\subsection{Servo}
V \LaTeX{}u je nový odstavec vytvořen jednoduše vložením jedné nebo více prázdných řádek. Rozestupy mezi odstavci, stejně jako zarovnání textu, můžeme upravit podle potřeby.


\subsection{Napájení}
V \LaTeX{}u je nový odstavec vytvořen jednoduše vložením jedné nebo více prázdných řádek. Rozestupy mezi odstavci, stejně jako zarovnání textu, můžeme upravit podle potřeby.

\section{Software}
\label{sec:prace_s_textem}

Tato část se zaměřuje na základní techniky práce s textem v \LaTeX{}u, včetně formátování textu, vytváření seznamů a využití křížových odkazů a poznámek pod čarou.

\subsection{WebServer}
Formátování textu je klíčovým prvkem pro zvýraznění důležitých informací a zlepšení čitelnosti dokumentu.

\subsection{Ovládání}
V \LaTeX{}u existuje několik způsobů, jak zvýraznit text. Můžeme použít tučné písmo, kurzívu nebo podtržení.

\begin{verbatim}
	\textbf{tučné písmo}, \textit{kurzíva}, \underline{podtržený text}
\end{verbatim}



	\chapter{Způsoby řešení a použité postupy}
	Jak už jsem psal výše \LaTeX je dosti komplexní systém, který umožňuje psát velmi rozsáhlé text. Jeho autor Donald Knuth ho stvořil, aby mohl vydat jeho učebnici \emph{The Art of Computer Programming} a dodnes se je využíván pro sazbu skript, učebnic, článků či závěrečných prací. V této kapitole najdeš ukázky různých funkcí a balíčků \LaTeX u od těch nejzákladnějších až po složitější. Neznamená to nutně, že všechny musíš použít, ale když potřebuješ pomoct, tak je dobré mít oporu. 
	
	Pokud s \LaTeX em úplně začínáš tak ti můžu doporučit přiručku \emph{Ne příliš stručný úvod do systému \LaTeX2e}~\cite{LaTeXprirucka}. Případně spoustu užitečných informací nalezneš na Wikibooks~\cite{wikibooksLaTeX}. Pokud narazíš na nějaký problém googli. Na internetu je spoustu fór, kde pravděpodobně už někdo podobný problém řešil. Asi nejvíce otho najdeš na stránce \emph{TeX - LaTeX Stackexchange} \cite{stackExchange}.
	
	
	\section{Sestavení modelu} %%[Text, který bude v obsahu]{Text, který se vytiskne na stránce} Zkus měnit jednotlivé závorky a uvidíš :) 
	Psaní v \LaTeX{u} není žádná věda, stačí psát normálně do zdrojového souboru. Pokud bys chtěl psát obrážky či číslovaný seznam, pak můžeš použít prostředí \texttt{itemize} či \texttt{enumerate}. Často je důležité používat nezlomitelnou mezeru. Tu uděláš pomocí \verb|~|~(tildy). Pokud budeš chtít psát uvozovky použij příkaz \texttt{uv}, pomocí něj se ti vytvoří uvozovky podle příslušného jazyka. V česku tedy ve formátu 99 66. Použití příkazu najdeš níže v textu.
	
	Občas je zapotřebí \LaTeX{u} pomoct při rozdělování slov. To se udělá snadno vložením symbolů \verb|\-| mezi jednotlivé slabiky.
 
    \section{Zprovoznění serveru} %%[Text, který bude v obsahu]{Text, který se vytiskne na stránce} Zkus měnit jednotlivé závorky a uvidíš :) 
	Psaní v \LaTeX{u} není žádná věda, stačí psát normálně do zdrojového souboru. Pokud bys chtěl psát obrážky či číslovaný seznam, pak můžeš použít prostředí \texttt{itemize} či \texttt{enumerate}. Často je důležité používat nezlomitelnou mezeru. Tu uděláš pomocí \verb|~|~(tildy). Pokud budeš chtít psát uvozovky použij příkaz \texttt{uv}, pomocí něj se ti vytvoří uvozovky podle příslušného jazyka. V česku tedy ve formátu 99 66. Použití příkazu najdeš níže v textu.
	
	Občas je zapotřebí \LaTeX{u} pomoct při rozdělování slov. To se udělá snadno vložením symbolů \verb|\-| mezi jednotlivé slabiky.
 
    \section{Kód pro ovládání tanku} %%[Text, který bude v obsahu]{Text, který se vytiskne na stránce} Zkus měnit jednotlivé závorky a uvidíš :) 
	Psaní v \LaTeX{u} není žádná věda, stačí psát normálně do zdrojového souboru. Pokud bys chtěl psát obrážky či číslovaný seznam, pak můžeš použít prostředí \texttt{itemize} či \texttt{enumerate}. Často je důležité používat nezlomitelnou mezeru. Tu uděláš pomocí \verb|~|~(tildy). Pokud budeš chtít psát uvozovky použij příkaz \texttt{uv}, pomocí něj se ti vytvoří uvozovky podle příslušného jazyka. V česku tedy ve formátu 99 66. Použití příkazu najdeš níže v textu.
	
	Občas je zapotřebí \LaTeX{u} pomoct při rozdělování slov. To se udělá snadno vložením symbolů \verb|\-| mezi jednotlivé slabiky.
	
	\subsection{Tabulky}
	
	U tabulek platí to stejné co u obrázků. Zarovnávají se na střed a nechávají se \uv{plavat} v textu. Tabulka narozdíl od textu, má popisek nahoře. U tabulky \ref{tab:ukazka} je použit balíček \texttt{booktabs}, pomocí kterého je celá tabulka naformátovaná.
	
	Seznam jak obrázků tak tabulek je pak vytvořen pomocí příkazů \texttt{listoftables} a~\texttt{list\-of\-fig\-ures} na konci práce před literaturou.
	
	\begin{table}[h]
		\caption{Tato tabulka slouží jako ukázka toho, jak mohou tabulky vypadat.} %% popisek se u tebaluky píše nad ní
		\label{tab:ukazka} %% označení pro pozdější odkazování se 
		\centering
		\begin{tabular}{lll}
			\toprule %% příkazy z balíčku booktabs
			záhlaví& této & tabulky\\
			\midrule
			obsah&tabulky& už\\
			není & oddělený &čarami\\
			\bottomrule
		\end{tabular}
	\end{table}
	
	
	\subsection{Obrázky}
	
	U obrázků je dobré používat vektorové formáty, pokud to jde. \LaTeX se nejvíc kamarádí s formátem PDF. Do známého PDFka lze z jiných vektorových formátů (ať už SVG či ESP) obrázky přenést snadno pomocí grafických programů, jako je třeba Inkscape. \LaTeX si rozhodně poradí i s tradičními formáty PNG a JPG, avšak tyto obrázky mohou zabírat více 



u a při tisku se může projevit nižší rozlišení obrázků. Pokud chceš používat tyto obrázky, rozhodně měj na paměti, aby měli rozlišení alespoň 250 indálně 330 ppi.
	
	Obrázky se vkládají do prostředí \texttt{figure}, při úpravě šířky je možné krom tradičních jednotek jako cm nebo mm použít také jako jednotku šířku stránky \texttt{textwidth} to se hodí zejména když chceš mít více podobrázků. 
	
	U každého obrázku je důležité aby měl popisek, \texttt{caption}. Do popisku napiš, co na obrázku je, případně nějaký další popis, tak aby čtenář následně neměl sebemenší pochybnost. U obrázků co nejsou tvoje nezapomeň an citaci. Jinak by to totiž znamenalo, že jsi obrázek dělal ty sám, což není etické přivlastňovat si cizí díla. Popisek obrázku je věta, proto musí vždy končit tečkou.
	
	\begin{figure}[h!]
		\centering %% příkaz, který ti obrázek zarovná na střed
		\includegraphics[width=0.6\textwidth]{image/logo-skoly.png} %% vložení samotného obrátku
		\caption{Logo SŠPU Opava \cite{sspuLogo}.} %% popisek obrázku, nezapomeň na citace!
		\label{fig:logoSSPU} %% označení až budeš chtít na obrázek odkazovat
	\end{figure}
	
	Když chceš odkazovat na obrázek, stačí pak už jen napsat příkaz \texttt{ref} a do závorek napsat označení obrázku. Třeba logo SOČky, můžeš vidět na obrázku \ref{fig:logoSSPU} \cite{socSSPU}.
	
	
	\begin{figure}[h!] \centering
		\begin{subfigure}[h]{0.63\textwidth} %%prostředí pro podobrázek {šířka podobrázku}
			\includegraphics[width=\textwidth]{image/vysledek_10} 
			\caption{} %% aby se ti vysázelo označení obrázku.
		\end{subfigure}
		\begin{subfigure}[h]{0.63\textwidth}
			\includegraphics[width=\textwidth]{image/vysledek_20}
			\caption{}
		\end{subfigure}
		\caption[Graf závislosti rotace DH PSF $\Delta\varphi$ na defokusaci objektivu $\Delta z$.]{Graf závislosti rotace DH PSF $\Delta\varphi$ na defokusaci objektivu $\Delta z$, (a) při použití objektivu Plan Fluor 10, (b) při použití objektivu Plan Fluor 20. Měřená data (žluté body) jsou lineárně proloženy (přerušovaná přímka). }
		\label{fig:rotace_grafy}
	\end{figure}
	
		Pokud bys měl více podobrázků přichází do hry balíček \texttt{subcaption}. Pomocí něj lze vysázet i podobrázky. U podobrázků se popisek píše pouze jeden, dolů. Je v tomto připadě vhodné použít navíc hranaté závorky, do nichž se napíše kratší popisek, který se následně ukáže v seznamu obrázků.
	
	Všimni si, že obrázky jsou naschvál široké. Je to proto, aby byly dobře čitelné. Také si všimni popisku grafů. Ačkoli nejspíš netušíš co je to DH PSF či defokusace objektivu mělo by ti být jasné, že je důležité přesně graf popsat. To znamená co je na vodorovné ose, co je na svislé ose. V jakých jednotkách veličiny jsou. Které body co znamenají, která křivka má jaký význam. Napsat samotné \uv{$\Delta \varphi$} je málo, vždy raději připoměň, co daná značka znamená.
	

	
	\subsection{Literatura}
	
	V \LaTeX{}u lze dělat seznam literatury dvěma způsoby. V této šabloně jsem použil ten, kdy se seznam literatury píše přímo do práce. Pro jeho vygenerování doporučuji použít některý z generátarů, jako jsou například Citace PRO \cite{citacePRO}. Pomocí citací lze vygenerovat přímo dokument, který se pak už jen překopíruje do textu a člověk nemusí nic zvýrazňovat. Dále lze využít Bibtex, který rozhodně do budoucna hodlám zaimplementovat do šablony, avšak jeho použití nemusí být tak přátelské k začátečníkům.
	
	Pokud bys chtěl odkazovat na vícero zdrojů stačí je napsat vedle sebe oddělené čárkou \cite{LaTeXprirucka, citacePRO, Born2019}. Případně můžu odkaz na konkrétní stránku dát do hranatých závorek, viz \cite[str.~1]{Born2019}
	
	\subsection{Programový kód}
	Pro vložení programového kódu do dokumentu LaTeX s možností zvýraznění syntaxe můžete použít balíček \texttt{listings}. Tento balíček nabízí široké možnosti pro formátování kódu, včetně zvýraznění syntaxe pro různé programovací jazyky.
	
	Nejprve je třeba do preambule LaTeX dokumentu přidat \texttt{\\usepackage{listings}} a nastavit příslušné parametry. Příklad nastavení pro jazyk Python by mohl vypadat takto:
	

\begin{lstlisting}[style=Python, caption={Ukázka Python kódu}]
	# Python code here
	def hello_world():
		print("Hello, world!")
\end{lstlisting}


\begin{lstlisting}[style=JavaScript, title={Kód}, caption={Ukázka JS kódu}]
	// JavaScript code here
	function helloWorld() {
		console.log("Hello, world!");
	}
\end{lstlisting}	
	
\begin{lstlisting}[style=ES6, caption={ES6 (ECMAScript-2015) Listing}]
	/* eslint-env es6 */
	/* eslint-disable no-unused-vars */
	
	import Axios from 'axios'
	import { BASE_URL } from './utils/api'
	import { getAPIToken } from './utils/helpers'
	
	export default class User {
		constructor () {
			this.id = null
			this.username = null
			this.email = ''
			this.isActive = false
			this.lastLogin = ''  // ISO 8601 formatted timestamp.
			this.lastPWChange = ''  // ISO 8601 formatted timestamp.
		}
	}
	
	const getUserProfile = async (id) => {
		let user = new User()
		await Axios.get(
		`${BASE_URL}/users/${id}`,
		{
			headers: {
				'Authorization': `Token ${getAPIToken()}`,
			}
		}
		).then{response => {
				// ...
			}).catch(error => {
				// ...
			})
		}
\end{lstlisting}	
	
	\chapter{Výsledky řešení}
	
	Každou práci je dobré zkontrolovat, aby v ní nebyly pravopisné chyby, nebyla těžkopádně napsaná -- byla čtivá -- a neobsahovala žádný typografický nedostatek. Proto, když práci sepíšeš, nech ji chvilku odležet, třeba týden. Pak si ji po sobě znovu přečti. Hned uvidíš, kolik věcí bys napsal jinak případně kde tě bije do očí jaká chyba. Dej práci přečíst také svému školiteli a případně češtináři. Zajistíš tak, že bude obsahovat méně chyb.
	
	Pak můžeš práci vytisknout a hurá do soutěže.
 
    \section{Podoba hardwarového zařízení}
    
    \section{Podoba webové aplikace}
    
    Webová aplikace spolehlivě splňuje svůj účel, uživatelé se připojí a monhou tank bez problému ovládat pomocí 4 směrových tlačítek a sledovat při tom živý obraz z kamery tanku. Aplikace je také plně responzivní, lze ji proto používat i na mobilních telefonech nebo tabletech. Do budoucna bych mohl zapracovat na vylepšeném vzhledu, přidat tmavý a světlý režim nebo podobné úpravy.
	
	\chapter*{Závěr}
	
    Celkově lze říci, že hlavním záměrem mého projektu bylo vytvořit plně funkční tank s možností dálkového ovládání prostřednictvím webové aplikace nebo ovladače. Většinu cílů, které jsem si stanovil se mi podařilo úspěšně splnit, z čehož mám osobně radost. Z časových důvodů se mi ale bohužel nepodařilo sestrojit ovladač, jako alternativní možnost ovládání. Tuto myšlenku avšak nemám v plánu zahodit a do budoucna bych na ní chtěl určitě ještě pracovat. Mezi mé další plány na vylepšení projektu patří zprovoznění otáčení věže, nebo další úpravy vzhledu. Celkově lze tedy říci, že tento projekt byl pro mě nejen technickou výzvou, ale i příležitostí k neustálému zdokonalování a rozšiřování mých znalostí.
	
	%% literatura
	\begin{thebibliography}{99}
		\bibitem{sablonaSOC} DOKULIL Jakub. \textit{Šablona pro psaní SOČ v programu \LaTeX} [Online]. Brno, 2020 [cit. 2020-08-24]. Dostupné z: \url{https://github.com/Kubiczek36/SOC_sablona}
		\bibitem{LaTeXprirucka}OETIKER, Tobias, Hubert PARTL, Irene HYNA, Elisabeth SCHEGL, Michal KOČER a Pavel SÝKORA. \textit{Ne příliš stručný úvod do systému LaTeX2e} [online]. 1998 [cit. 2020-08-24]. Dostupné z: \url{https://www.jaroska.cz/elearning/informatika/typografie/lshort2e-cz.pdf}
		\bibitem{wikibooksLaTeX}\textit{Wikibooks: LaTeX} [online]. San Francisco (CA): Wikimedia Foundation, 2001- [cit. 2020-08-24]. Dostupné z: \url{https://en.wikibooks.org/wiki/LaTeX}
		\bibitem{stackExchange} \textit{TeX - LaTeX Stack Exchange} [online]. Stack Exchange, 2020 [cit. 2020-09-01]. Dostupné z: \url{https://tex.stackexchange.com}
		\bibitem{sspuLogo} \textit{Střední škola průmyslová a umělecká Opava} [online]. [cit. 2023-11-11]. Dostupné z: \url{https://www.sspu-opava.cz}
		\bibitem{citacePRO}\textit{Citace PRO} [online]. Citace.com, 2020 [cit. 2020-08-31]. Dostupné z: \url{https://www.citacepro.com}
		\bibitem{Born2019} BORN, Max a Emil WOLF. \textit{Principles of optics: electromagnetic theory of propagation, interference and diffraction of light}. 7th (expanded) edition. Reprinted wirth corrections 2002. 15th printing 2019. Cambridge: Cambridge University Press, 2019. ISBN 978-0-521-64222-4.
	\end{thebibliography}
	
	%% obrázky 
	\listoffigures
	
	%% tabulky
	\listoftables
	
	\appendix %% začínají přílohy
	
	\titleformat{\chapter}[block]{\scshape\bfseries\LARGE}{Příloha \thechapter}{10pt}{\vspace{0pt}}[\vspace{-22pt}] %% nastavení nadpisu u příloh

\end{document}